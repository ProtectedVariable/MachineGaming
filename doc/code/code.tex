\documentclass{article}
\usepackage[utf8]{inputenc}
\setlength{\parindent}{0cm}
\addtolength{\hoffset}{-2cm}
\addtolength{\textwidth}{4cm}
\usepackage[frenchb]{babel}
\usepackage[T1]{fontenc}
\usepackage[hidelinks]{hyperref}
\usepackage{graphicx}
\usepackage{afterpage}
\usepackage{minted}
\usepackage{pdflscape}
\usepackage{pdfpages}
\usepackage{amssymb}
\usepackage{amsmath}
\usepackage{float}
\usepackage{fancyhdr}

\pagestyle{fancy}
\fancyhf{}
\fancyhead[LE,RO]{\leftmark}
\fancyfoot[R]{\thepage}
\fancyfoot[L]{Ibanez Thomas}

\newcommand*{\printdate}{%
   \ifcase \month\or Janvier\or Février\or Mars\or Avril\or Mai\or Juin\or Juillet\or Août\or Septembre\or Octobre\or Novembre\or Décembre\fi \space \number\year}

\title{Algorithmes génétiques pour l'intelligence artificielle - Annexe 1 - Code source}
\author{Thomas Ibanez}
\makeindex


\begin{document}

\begin{titlepage}
	\vspace*{-3cm}
		\includegraphics[width=.3\linewidth]{hepia.png}
		\hfill
		\includegraphics[width=.3\linewidth]{hes.png}\par
	\vspace{1cm}
	\centering
	{\scshape\huge Algorithmes génétiques pour l'intelligence artificielle - Annexe 1 - Code source \par}
	\vspace{1cm}
	\includegraphics[scale=0.4]{logo.png}\par
	\vspace{1.5cm}
	{\Large\scshape Thèse de bachelor présentée par\par}
	\vspace{0.5cm}
	{\bfseries\Large M. Thomas Ibanez\par}
	\vspace{0.8cm}
	{\scshape\Large Pour l'obtention du titre Bachelor of Science HES-SO en\par}
	\vspace{0.5cm}
	{\bfseries\Large Ingénierie des technologies de l'information avec orientation en Logiciels et Systèmes complexes\par}
	\vfill
	Professeur HES responsable\par
	\bfseries{Orestis Malaspinas\par}

	\vfill

	{\large\ Septembre 2018 \par}
\end{titlepage}

\tableofcontents

\newpage
\section{Installation}

Les outils suivants doivent être présent sur votre machine pour installer les différents composants du projet:

\begin{itemize}
	\item \textbf{protoc} Compilateur protobuf, téléchargeable depuis \url{https://github.com/google/protobuf/releases}
	\item \textbf{java} JDK version 1.7+ et JRE 7
	\item \textbf{make}
	\item \textbf{node.js} téléchargeable depuis \url{https://nodejs.org/en/}
\end{itemize}

A partir du dossier racine du projet (clonable depuis \url{https://github.com/ProtectedVariable/MachineGaming.git}), exécutez la commande make qui va se charger de la compilation.\\

Pour démarrer le serveur allez dans le dossier \textbf{server} et entrez la commande \textbf{npm install}, cette commande va démarrer le téléchargement des dépendances nécessaire au fonctionnement du serveur. Une fois l'installation terminée, tapez \textbf{npm start} pour démarrer le serveur.\\

Pour démarrer le client ouvrez le dossier \textbf{client} dans un IDE tel qu'Eclipse et compilez le.


\newpage
\section{Serveur}

\inputminted[breaklines,breaksymbol=, frame=single,label=server.js, stepnumber=1,tabsize=2]{javascript}{server.js}

\inputminted[breaklines,breaksymbol=, frame=single,label=pool.js, stepnumber=1,tabsize=2]{javascript}{pool.js}

\inputminted[breaklines,breaksymbol=, frame=single,label=genetic.js, stepnumber=1,tabsize=2]{javascript}{genetic.js}

\inputminted[breaklines,breaksymbol=, frame=single,label=client.js, stepnumber=1,tabsize=2]{javascript}{client.js}

\inputminted[breaklines,breaksymbol=, frame=single,label=mgnetwork.js, stepnumber=1,tabsize=2]{javascript}{mgnetwork.js}

\inputminted[breaklines,breaksymbol=, frame=single,label=mlp.js, stepnumber=1,tabsize=2]{javascript}{mlp.js}

\inputminted[breaklines,breaksymbol=, frame=single,label=neat.js, stepnumber=1,tabsize=2]{javascript}{neat.js}

\newpage
\section{Client}

\inputminted[breaklines,breaksymbol=, frame=single,label=Client.java, stepnumber=1,tabsize=2]{java}{Client.java}

\inputminted[breaklines,breaksymbol=, frame=single,label=GenomeCodec.java, stepnumber=1,tabsize=2]{java}{GenomeCodec.java}

\inputminted[breaklines,breaksymbol=, frame=single,label=Network.java, stepnumber=1,tabsize=2]{java}{Network.java}

\inputminted[breaklines,breaksymbol=, frame=single,label=ActivationFunction.java, stepnumber=1,tabsize=2]{java}{ActivationFunction.java}

\inputminted[breaklines,breaksymbol=, frame=single,label=ActivationFunctions.java, stepnumber=1,tabsize=2]{java}{ActivationFunctions.java}

\inputminted[breaklines,breaksymbol=, frame=single,label=MultilayerPerceptron.java, stepnumber=1,tabsize=2]{java}{MultilayerPerceptron.java}

\inputminted[breaklines,breaksymbol=, frame=single,label=NEATNetwork.java, stepnumber=1,tabsize=2]{java}{NEATNetwork.java}

\inputminted[breaklines,breaksymbol=, frame=single,label=NeuralNetwork.java, stepnumber=1,tabsize=2]{java}{NeuralNetwork.java}

\inputminted[breaklines,breaksymbol=, frame=single,label=AsteroidSimulator.java, stepnumber=1,tabsize=2]{java}{AsteroidSimulator.java}

\inputminted[breaklines,breaksymbol=, frame=single,label=Display.java, stepnumber=1,tabsize=2]{java}{Display.java}

\inputminted[breaklines,breaksymbol=, frame=single,label=Simulator.java, stepnumber=1,tabsize=2]{java}{Simulator.java}

\newpage
\section{Protocole}

\inputminted[breaklines,breaksymbol=, frame=single,label=mg.proto, stepnumber=1,tabsize=2]{protobuf}{mg.proto}


\end{document}
